\documentclass{scrartcl}	    % Klasse für Artikel
\usepackage[ngerman]{babel}	% sprachspezifische Übersetzungen
\usepackage[T1]{fontenc}	    % Umlaute
\usepackage[utf8]{inputenc}	% Unterstützung erweiterter Eingabe-Zeichensätze
\usepackage{amsmath}		    % Erweiterungen für mathematischen Formelsatz
\usepackage{graphicx}        % Einbinden von Bildern
\usepackage{array}           % vertikale Ausrichtung in Tabellen

\parindent 0mm               % kein Einzug am Absatzbeginn
\pagestyle{empty}            % keine Seitennummer

\def\uebung{10}              % Definition einer Variable (uebung)
                             % und Zuweisung eines Wertes (10)

\begin{document}

\begin{table}[h]\Large
\begin{center}
\begin{tabular}{|c|m{11.5cm}|c|}
\hline
EP\&P
&
Übungen Programmierung\newline
Dr. J. Brose, PHY C116, Tel. 32104\newline
J.Brose@physik.tu-dresden.de
&
\uebung
\\
\hline
\end{tabular}
\end{center}
\end{table}
               % Einbinden eines weiteren Dokuments

\section*{Numerische Nullstellensuche: Keplerproblem}


\textit{Hinweise}:\\

Der oben angegebene Kopf der Titelseite (zu finden in
\texttt{header.tex}) sollte so verwendet werden.\\

Die reinen Textbestandteile lassen sich mit Hilfe eines
pdf-Betrachters aus der Aufgaben-Datei \texttt{U10.pdf} kopieren.\\

Die in der Aufgabenstellung der 10. Übung verwendeten geschweiften
Klammer unter Formelteilen lassen sich im Mathematik-Modus mittels
{\verb \underbrace } erzeugen, z.\,B.:

\[
n^2 - \underbrace{(n-1)^2}_{\text{binom. Formel}} = 2n-1
\]

\end{document}
