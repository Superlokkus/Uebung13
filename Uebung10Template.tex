\documentclass{scrartcl}	    % Klasse für Artikel
\usepackage[utf8]{inputenc}	% Unterstützung erweiterter Eingabe-Zeichensätze
\usepackage[ngerman]{babel}	% sprachspezifische √úbersetzungen
\usepackage[T1]{fontenc}	    % Umlaute
\usepackage{amsmath}		    % Erweiterungen für mathematischen Formelsatz
\usepackage{graphicx}        % Einbinden von Bildern
\usepackage{array}           % vertikale Ausrichtung in Tabellen
\usepackage{wrapfig}
\usepackage{hyperref}


\parindent 0mm               % kein Einzug am Absatzbeginn
\pagestyle{empty}            % keine Seitennummer

\def\uebung{10}              % Definition einer Variable (uebung)
                             % und Zuweisung eines Wertes (10)

\begin{document}

\begin{table}[h]\Large
\begin{center}
\begin{tabular}{|c|m{11.5cm}|c|}
\hline
EP\&P
&
Übungen Programmierung\newline
Dr. J. Brose, PHY C116, Tel. 32104\newline
J.Brose@physik.tu-dresden.de
&
\uebung
\\
\hline
\end{tabular}
\end{center}
\end{table}
               % Einbinden eines weiteren Dokuments

\section*{Numerische Nullstellensuche: Keplerproblem}
\emph{Die Bahnen der Planeten um die Sonne sind Ellipsen, in deren einem Brennpunkt die Sonne steht.}

Dieser einfache Sachverhalt führt zu einem analytisch nicht lösbaren Problem, wenn man versucht im System der Sonne, den Betrag des Ortsvektors $\vec{r}$(t) oder auch den Winkel $\phi$(t) des Planeten relativ zur Sonne im Zeitverlauf zu berechnen.

\begin{wrapfigure}{l}{0.5\textwidth}
\includegraphics[width=0.48\textwidth]{Kepler_Schema.png}
\end{wrapfigure}

Die Orts- Zeitfunktion r(t) lässt sich über
die Zeitabhängigkeit der \textbf{wahren Anomalie
$\boldsymbol{\phi}$(t)} berechnen, wobei $\boldsymbol{\phi}$\textbf{(t)} der Winkel zwischen der Verbindungslinie Sonne - Perihel und der Linie Sonne - Planet ist.
\[
\mathbf{ r = r(\boldsymbol{\phi} (t)) = r(t) = a \cdot \frac{1-\boldsymbol{\epsilon}^2}{1+\boldsymbol{\epsilon} cos \boldsymbol{\phi}}}
\]

mit

\begin{description}
\item $\boldsymbol{\phi}$ -- wahre Anommalie
\item $\boldsymbol{\epsilon}$ -- numerische Exzentrität
\item $\boldsymbol{\epsilon}$ = $\frac{e}{a} = \sqrt{1-{(\frac{b}{a})}^2}$
\item \textbf{e} -- lineare Exzentrizität
\item \textbf{e} = $\sqrt{a^2 - b^2}$
\item \textbf{a} -- große Halbachse der Bahn-Ellipse
\item \textbf{b} -- kleine Halbachse der Bahn-Ellipse
\end{description}
Die wahre Anomalie ist berechenbar aus der \textbf{exzentrischen Anomalie E} (Winkel vom Mittelpunkt der Ellipse zu einem Hilfspunkt auf dem Umkreis mit Radius der großen Halbachse (Projektion des Planetenortes auf den Umkreis) relativ zur Verbindungslinie Mittelpunkt - Perihel)

\[
\phi = \underbrace{2\pi - \underbrace{arccos(\frac{cos E-\epsilon}{1-\epsilon cos E})}_{0 \le E \le \pi}}_{\pi \le E \le 2\pi}
\]
Die exzentrische Anomalie E ergibt sich aus der analytisch nicht löbaren \emph{Kepler-Gleichung}

\[
E - \epsilon sin E = 2 \pi \frac{t - t_0}{T}
\]
\begin{description}
\item $T$ -- Bahnperiode
\item $t_0$ -- Perihelzeit, Zeit des Erreichens des Sonnen-nächsten Punktes
\end{description}

\section*{Aufgaben}
Verwenden Sie die in \emph{"`/home/data/Programmierung/Uebung10/kepler\_template.py"'} 
vorgegebenen Funktionsdeklarationen und das Dictionary mit den Planetendaten als Grundlage für ihr Skript!

\begin{enumerate}
\item Schreiben Sie eine Funktion, die mit Hilfe des Newtonschen Näherungsverfahrens die exzentrische Anomalie zu beliebigen Zeiten während eines vollständigen Umlaufs (startend im Perihel) für elliptische Bahnen um die Sonne berechnet!
\item Berechnen Sie in geeigneten Zeitabständen mit Hilfe der exzentrischen Anomalie jeweils die wahre Anomalie und den Abstand des Himmelskörpers von der Sonne für die Erde, Pluto und den Halley\'schen Kometen während eines Umlaufs und stellen Sie diese in \href{http://matplotlib.org/api/pyplot_api.html#matplotlib.pyplot.polar}{Polarkoordinaten} grafisch dar! %Ich weigere mich den Link in diesem IE6 Blau zu machen

\begin{tabular}{|m{1cm}|m{3.5cm}|m{3.5cm}|m{3.5cm}|}\hline
&Umlaufperiode \newline T& große Halbachse \newline a [AE]&numerische Exzentrizität $\epsilon$\\ \hline
Erde&365.25 d&1.0&0.0167\\
Pluto&248 a&39.44&0.25\\
Halley&76 a&17.94&0.97\\ \hline
\end{tabular}
\end{enumerate}

\end{document}
